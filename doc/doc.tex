\documentclass[a4paper, 12pt, slovak]{article}
\usepackage[czech]{babel}
\usepackage[utf8]{inputenc}
\usepackage{ifpdf}
\usepackage{url}
\usepackage{color}
\usepackage{float}
\usepackage{listings}
\usepackage{graphicx}
\usepackage{amsfonts}
\usepackage{epstopdf}
\usepackage[top=50pt,bottom=50pt,left=50pt,right=50pt]{geometry}

\lstdefinestyle{c++}
{
    language=C++,
    frame=none,
    basicstyle=\footnotesize\ttfamily,
    showspaces=false,
    keywordstyle=\color{blue},
    commentstyle=\color{red}
    texcl=false,
    mathescape=false,
    prebreak = \raisebox{0ex}[0ex][0ex]{\ensuremath{\hookleftarrow}}
}
\lstnewenvironment{c++}
{\lstset{style=c++}}
{}

\renewcommand{\contentsname}{Obsah}

\newtheorem{definicia}{Definícia}

\begin{document}

\title{Manuál k aplikácií a knižnici cheetah}
\date{}
\author{Peter Mitura}
\maketitle

\emph{Cheetah} je aplikácia a knižnica umožňujúca efektívne riešiť problém konvexnej 
obálky. Tento manuál popisuje všetky možnosti aplikácie, funkcie knižnice a postup 
ich inštalácie.

\tableofcontents
\pagebreak

\section{Inštalácia}
Popis inštalačných procedúr, potrebných k sprevádzkovaniu projektu \emph{Cheetah} na 
vašom počítači.
\subsection{Požiadavky}
Projekt je určený pre systém GNU/Linux, prípadne iné unix-like systémy. Pre 
inštaláciu je ďalej potrebné, aby ste na systéme mali nainštalované nasledovné 
komponenty:

\begin{itemize}
 \item \textbf{GCC} verzia 5.0 alebo vyššia, s podporou OpenMP 
(\url{https://gcc.gnu.org/})
 \item \textbf{GNU ar} (\url{https://sourceware.org/binutils/docs/binutils/ar.html})
 \item \textbf{GNU Make} (\url{https://www.gnu.org/software/make/})
 \item \emph{(voliteľne)} \textbf{Google Test} 
(\url{https://github.com/google/googletest})
\end{itemize}

Google Test je potrebný len k sprevádzkovaniu modulu jednotkových testov, aplikácia 
aj knižnica môže fungovať aj bez neho.

\subsection{Kompilácia}
Kompilácia je automatizovaná nástrojom GNU Make, na výber sú tri direktívy, ktorými 
sa kompilujú rôzne moduly projektu:

\begin{itemize}
 \item \texttt{make}, \texttt{make app} -- Aplikácia pre príkazový riadok, vytvorený 
je spustiteľný súbor \texttt{cheetah} v~adresári \texttt{bin}
 \item \texttt{make lib} -- Knižnica, do adresára \texttt{bin} je skompilovaný súbor 
\texttt{libcheetah.a} a vytvorený adresár \texttt{include}, obsahujúci príslušné 
hlavičkové súbory.
 \item \texttt{make test} -- Jednotkové testy, po kompilácií sú automaticky aj 
spustené a vyhodnotené.
\end{itemize}

Upozorňujeme, že každý z týchto modulov používa zdieľané objektové súbory v adresári 
\texttt{build}. Je preto nutné pri každej zmene modulu zavolať príkaz \texttt{make 
clean}, ktorý vráti projekt do počiatočného stavu.

\section{Aplikácia}
Aplikácia sa po kompilácií nachádza v adresári \texttt{bin} pod názvom 
\texttt{cheetah}, je možné ju ďalej skopírovať do jedného z adresárov definovaných
v~premennej prostredia \texttt{PATH} a používať z ľubo\-voľ\-ného miesta v systéme.

\subsection{Formát vstupu}
Program prijíma textový vstup, začínajúci prirodzeným číslom $n$, označujúcim počet 
vstupných bodov, za ktorým nasleduje $n$ dvojíc alebo trojíc reálnych čísel, 
reprezentujúcich súradnice bodov v $\mathbb{R}^2$ resp. $\mathbb{R}^3$. Čísla 
sú oddelené bielymi znakmi. Príklad vstupu v $\mathbb{R}^2$:

\begin{c++}
4
0 10
-10 0
9.8002 0
0 1
\end{c++}

\subsection{Formát výstupu}
Výstup v $\mathbb{R}^2$ má rovnakú podobu, ako vstup. Body v ňom sú vypísané v 
poradí, v ktorom sa objavujú na obálke, teda proti smeru hodinových ručičiek. 

Výstup v $\mathbb{R}^3$ je odlišný, začína sa prirodzeným číslom $n$, označujúcim 
počet stien. Za ním nasleduje popis $n$ stien, pri čom popis jednej steny sa skladá 
z prirodzeného čísla $m$ a po ňom $m$ trojíc reálnych čísel, predstavujúcich 
súradnice vrcholov na danej stene. Vrcholy v rámci jednej steny sú zoradené proti 
smeru hodinových ručičiek

\subsection{Základné použitie}
Bez špecifikovaných prepínačov aplikácia hľadá konvexnné obálky v $\mathbb{R}^2$ s 
použitím algoritmu Quickhull. Vstupné dáta sú očakávané na štandardnom vstupe, výstup 
je vypísaný na štandardný výstup. Ak by sme vyššie uvedený príklad vstupu uložili do 
súboru \texttt{data.in}, použitie by mohlo vyzerať nasledovne:

\begin{c++}
$ bin/cheetah < data.in
3
9.8002 0
-10 0
0 10
\end{c++}

Funkčnosť je ďalej možné rozšíriť prepínačmi, ktoré popíšeme v nasledujúcich 
sekciach.

\subsection{Vstup zo súboru}
Prepínač \texttt{-i <názov súboru>} umožní načítať vstup z určeného súboru. Príklad 
použitia:

\begin{c++}
$ bin/cheetah -i data.in
3
9.8002 0
-10 0
0 10
\end{c++}

\subsection{Výstup do súboru}
Prepínač \texttt{-o <názov súboru>} presmeruje výstup zo štandarného výstupu do 
určeného súboru. Príklad použitia:

\begin{c++}
$ bin/cheetah -i data.in -o data.out
$ cat data.out
3
9.8002 0
-10 0
0 10
\end{c++}

\subsection{Meranie času}
Prepínač \texttt{-t} na štandardný výstup čas, ktorý zaberie nájdenie obálky. Do 
času sa nezarátava doba potrebná k načítaniu vstupu, alebo vypísaniu výstupu. 
Príklad použitia:

\begin{c++}
$ bin/cheetah -i data.in -o data.out -t
Execution time: 0.000160927 s.
\end{c++}
\pagebreak

\subsection{Použitie konkrétneho algoritmu}
Štandardným algoritmom pre riešenie je Quickhull. Na zvolenie iného algoritmu pre 
riešenie je možné použiť prepínač \texttt{-s <názov algoritmu>}. Dostupné voľby v 
$\mathbb{R}^2$ a ich asymptotické zložitosti sú:

\begin{itemize}
 \item \texttt{jarvis} -- Jarvis March ($\mathcal{O}(nh)$)
 \item \texttt{graham} -- Graham Scan ($\mathcal{O}(n\log n)$)
 \item \texttt{quickhull} -- Quickhull ($\mathcal{O}(nh)$ najhorší prípad, 
$\mathcal{O}(n\log n)$ v priemere)
 \item \texttt{chan} -- Chanov algoritmus ($\mathcal{O}(n\log h)$)
 \item \texttt{andrew} -- (\emph{experimentálne}) Andrewov algoritmus 
($\mathcal{O}(n\log n)$)
\end{itemize}

Pre $\mathbb{R}^3$ je dostupný len algoritmus Jarvis March, ktorý je použitý aj v 
základe.

\subsection{Voľba dimenzie}
V základe program používa rovinné algoritmy, dimenziu vstupu je možné špecifikovať 
prepínačom \texttt{-d <dimenzia>}. Súčasne podporované hodnoty sú 2 a 3.

\subsection{Paralelizácia}
Aplikácia štandardne púšťa sekvenčné verzie algoritmov, paralelizáciu je možné 
zapnúť prepína\-čom \texttt{-m <počet vlákien>}. Funguje len v $\mathbb{R}^2$ a 
maximálny počet vlákien je z dôvodu ochrany stability systému limitovaný na 24. 
Príklad použitia:

\begin{c++}
$ bin/cheetah -t -o /dev/null -i bigdata.in -s graham
Execution time: 0.700449 s.
$ bin/cheetah -t -o /dev/null -i bigdata.in -s graham -m 2
Execution time: 0.467592 s.
\end{c++}

\subsection{Test výkonu}
Výkon je mimo prepínača \texttt{-t} možné otestovať aj priloženým generátorom dát, 
ktorý dokáže vytvoriť vstup s daným počtom bodov vo vstupnej množine a na obálke. 
Test sa spúšťa prepínačom \texttt{-p <n> <h> <s> <r> <t>}, kde $n$ je počet bodov vo 
vstupnej množine ($n \in \mathbb{N}, n < 2^{31}$), $h$ počet bodov na obálke ($h 
\in \mathbb{N}, h \leq n$), $s$ určuje rozsah súradníc bodov $[-s, s]$ ($s \in 
\mathbb{N}$), $r$ počet zopakovaní testu ($r \in \mathbb{N}$) (čas sa ráta ako súčet 
všetkých behov) a $t$ počet vlákien ($t \in \mathbb{N}, 1 \leq t \leq 24$).

Upozorňujeme, že pri veľkom množstve bodov na obálke (typicky nad $5\cdot10^4$) sa 
začne prejavovať nepresnosť dátového typu \texttt{double} a na nájdených obálkach 
sa začne objavovať nižší počet bodov ako bol zadaný, čo môže ovplyvniť aj časy 
výpočtu. Testovač na túto skutočnosť upozorní varovaním. Príklad:

\begin{c++}
$ bin/cheetah -p 10000000 10 1000 1 1
Generating test instance...
...done, running test
time: 0.586616 s
$ bin/cheetah -p 10000000 10000 1000 1 1
Generating test instance...
...done, running test
time: 1.82755 s
$ bin/cheetah -p 10000000 1000000 1000 1 1
Generating test instance...
...done, running test
[WARNING] Precision errors occured (h may be too high)
time: 2.60803 s
\end{c++}

\section{Knižnica}
Modul knižnice umožňuje pridať implementované algoritmy priamo do programov 
napísaných v jazyku C++. Oproti aplikácií v nej chýbajú výskumne založené 
vymoženosti ako testovanie rýchlosti, obsahuje však plné možnosti ohľadne výberu 
algoritmov a paralelizácie.

\subsection{Použitie}
Knižnica je kompilovaná pre statické linkovanie s cieľovou aplikáciou. Pri 
kompilácií výsledku je nutné použiť minimálne prepínače \texttt{-std=c++11} a 
\texttt{-fopenmp}. Ďalej je nutné direktívami \texttt{-I} a \texttt{-L} špecifikovať 
cestu k hlavičkovým súborom a knižnici a prepínačom \texttt{-lcheetah} povoliť 
linkovanie. Funkcie knižnice sa nachádzajú v hlavičkovom súbore 
\texttt{cheetah/core.h}

K lepšiemu pochopeniu poslúži ukážková aplikácia využívajúca našu knižnicu. Do 
koreňového adresára najprv vložíme skompilovaný súbor \texttt{libcheetah.a} a 
adresár {include}. Potom vytvoríme nasledovný súbor \texttt{main.cpp}:

\begin{c++}
#include <iostream>
#include "cheetah/core.h"

int main() {
    ch::Points2D input, output;
    input.add({10, 0});
    input.add({-10, 0});
    input.add({0, 10});
    input.add({1, 1});
    ch::findHull(input, output);
    std::cout << output.getSize() << std::endl;
    return 0;
}
\end{c++}

\emph{(Fungovanie jednotlivých funkcií a dátových štruktúr vysvetlíme ďalej v tejto 
sekcií)}

Kompiláciu a spustenie potom môžeme realizovať nasledovnou sekvenciou príkazov:

\begin{c++}
$ g++ -std=c++11 -fopenmp -I include/ -L . main.cpp -lcheetah -o sample
$ ./sample 
3
\end{c++}

\subsection{Dátové štruktúry}
Pre reprezentáciu množín bodov sú použité dátové štruktúry \texttt{ch::Points2D} a 
\texttt{ch::Points3D}. Obe disponujú metódou

\begin{c++}
const std::vector<std::vector<double>>& getData() const;
\end{c++}

ktorá vráti konštantnú referenciu na ich vnútornú dátovú reprezentáciu, z ktorej je 
možné čítať ich obsah. Pre pridávanie bodov slúži funkcia:

\begin{c++}
bool add(std::vector<double> point);
\end{c++}

ktorá vráti \texttt{false} v prípade, že sa počet súradníc v posielanom vektore 
nezhoduje s počtom dimenzií danej množiny (a daný bod potom nie je ani pridaný do 
množiny).

Nakoniec majú ešte obe štruktúry metódu \texttt{getSize}, ktorá vŕati počet bodov v 
nich. 

Pre reprentáciu konvexného mnohostena, ktorý je výstupom hľadania obálky v troch 
rozmerov je štruktúra \texttt{ch::Polyhedron}. Tá je reprezentovaná ako 
\texttt{std::vector<ch::Points3D>}, teda ako vektor stien, ktoré sú uložené ako 
zoznam vrcholov. K tejto reprezentácií sa analogicky dá pristúpiť metódou 
\texttt{getFaces}, pre vkladanie bodov slúži metóda \texttt{addFace(ch::Points3D)}.

\subsection{Funkcie}
V tejto sekcií uvedieme zoznam funkcií, ktoré sprístupňuje hlavičkový súbor 
\texttt{core.h} a popis ich použitia. Všetky funkcie sú súčasťou menného priestoru 
\textit{ch} a je ich potrebné volať so zodpovedajúcou predponou.

\subsubsection{Nájdenie obálky}
Nájdenie obálky v $\mathbb{R}^2$ pomocou algoritmu Quickhull.

\begin{c++}
Points2D& findHull(const Points2D& input, Points2D& output);
\end{c++}

\begin {itemize}
 \item \textbf{parameter \textit{input}} -- Referencia na vstupnú sadu bodov. 
 \item \textbf{parameter \textit{output}} -- Referencia na výstupnú dátovú 
štruktúru, do ktorej bude vložený zoznam bodov tvoriacich obálku, zoradenú proti 
smeru hodinových ručičiek.
 \item \textbf{návratová hodnota} -- Rovnaká ako \texttt{output}.
\end {itemize}

\subsubsection{Nájdenie obálky s voľbou algoritmu}

Nájdenie obálky v $\mathbb{R}^2$ pomocou zvoleného algoritmu:

\begin{c++}
Points2D& findHull(const Points2D& input, Points2D& output, SolverType type);
\end{c++}

\begin{itemize}
 \item \textbf{parameter \textit{input}} -- Referencia na vstupnú sadu bodov. 
 \item \textbf{parameter \textit{output}} -- Referencia na výstupnú dátovú 
štruktúru, do ktorej bude vložený zoznam bodov tvoriacich obálku, zoradenú proti 
smeru hodinových ručičiek.
 \item \textbf{parameter \textit{input}} -- Označenie zvoleného algoritmu. Dostupné 
voľby:
\begin{itemize}
 \item \texttt{ch::JARVIS}
 \item \texttt{ch::GRAHAM}
 \item \texttt{ch::QUICKHULL}
 \item \texttt{ch::CHAN}
 \item \texttt{ch::ANDREW} \emph{(experimentálny)}
\end {itemize}
 \item \textbf{návratová hodnota} -- Rovnaká ako \texttt{output}.

\end {itemize}

\subsubsection{Paralelné nájdenie obálky}
Pre paralelné nájdenie obálky sú dostupné dve funkcie, analogické s predošlou 
dvojicou:

\begin{c++}
Points2D& findHullParallel(const Points2D& input, Points2D& output, int thr);

Points2D& findHullParallel(const Points2D& input, Points2D& output, 
    SolverType type, int thr);
\end{c++}

Jediný nový parameter je prirodzené číslo \texttt{thr}, ktoré určuje počet použitých 
vlákien (maximálne ale 24).
\pagebreak

\subsubsection{Nájdenie obálky v 3D}

Nájdenie obálky v $\mathbb{R}^3$ pomocou algoritmu Jarvis March:

\begin{c++}
Polyhedron& findHull3D(const Points3D input, Polyhedron& output)
\end{c++}

\begin{itemize}
 \item \textbf{parameter \textit{input}} -- Referencia na vstupnú sadu 3D bodov. 
 \item \textbf{parameter \textit{output}} -- Referencia na výstupnú dátovú 
štruktúru, reprezentujúcu konvexný mnohosten.
 \item \textbf{návratová hodnota} -- Rovnaká ako \texttt{output}.
\end{itemize}

\subsubsection{Aproximácia obálky (experimentálne)}
Do knižnice sme experimentálne zaradili aj rýchlu aproximáciu konvexnej obálky 
algoritmom \emph{BFP}. Upozorňujeme teda, že aj keď je tento algoritmus 
najrýchlejší, správnosť výsledku nie je očakávaná a je bude sa riešeniu len do istej 
miery blížiť.

\begin{c++}
Points2D& approximateHull(const Points2D& input, Points2D& output)
\end{c++}

\begin{itemize}
 \item \textbf{parameter \textit{input}} -- Referencia na vstupnú sadu bodov. 
 \item \textbf{parameter \textit{output}} -- Referencia na výstupnú dátovú 
štruktúru, obsahujúcu približnú podobu obálky.
 \item \textbf{návratová hodnota} -- Rovnaká ako \texttt{output}.
\end{itemize}


\end{document}
